\documentclass[a4paper,11pt,twoside]{book}			
\usepackage[italian]{babel}			
\usepackage{lipsum}								
\usepackage{listings}							
\usepackage{url}								
\usepackage{graphicx}							
\usepackage{geometry}						
\usepackage[dvipsnames]{xcolor} 				
\usepackage[hidelinks]{hyperref}		
\usepackage{chngpage}
\usepackage{multicol}

\usepackage{authblk}
\renewcommand\Authand{ e }

		
\geometry{a4paper,top=2cm,bottom=2cm,left=3cm,right=3cm,heightrounded,bindingoffset=10mm}
\raggedbottom									

\lstset{language=Java,						
	showspaces=false,
	showtabs=false,
	breaklines=true,
	showstringspaces=false,
	breakatwhitespace=true,
	commentstyle=\color{ForestGreen},
	keywordstyle=\color{blue},
	stringstyle=\color{red},
	identifierstyle=\color{Gray},
	basicstyle=\small\ttfamily
}


\begin{document}

\title{Interazione Uomo Macchina per Informatici}
\author[1]{Daniele Mazzei, Dipartimento di Informatica Università di Pisa}
\date{a.a. 2020/2021}
\maketitle

\tableofcontents


\chapter{Introduzione}

Negli ultimi anni il ruolo degli informatici \`e decisamente cambiato. Per anni l'informatico \`e stato chiuso in uno sgabuzzino ad interagire in
solitudine con una tastiera, bevendo bibite gassate mentre qualcuno gli diceva che cosa serviva (o almeno era convinto di sapere cosa servisse)
all'azienda per crescere.

Poi sono arrivati Internet, gli smartphone, la banda larga e l'Internet delle cose (IOT) e tutto \`e cambiato. 

Oggi gli informatici sono diventati le nuove star! 

Il mondo \`e ormai tecnocratico e le nuove soluzioni informatiche e tecnologiche hanno la capacit\`a di mutare la vita delle persone e gli andamenti
dell'economia in tempi così veloci da far rabbrividire.

Tuttavia, l'informatico \`e rimasto (spesso) in termini di attitudine e di bagaglio culturale lo stesso di prima. Li abbiamo fatti uscire dagli
sgabuzzini, abbiamo messo loro una giacca sopra la maglietta di Star Wars e li abbiamo spediti sui palchi dei tecnoeventi a fare pitch.

\`E chiaro che le competenze tecniche siano il bagaglio fondamentale per un informatico, ma in un epoca in cui le scelte degli informatici hanno
la potenzialit\`a di cambiare la vita delle persone non si può più prescindere da far capire agli informatici che cosa accade ad una persona ``normale"
(un non informatico, un babbano) quando interagisce con un software o con un sistema tecnologico in generale. 

Per troppi anni gli informatici hanno potuto limitarsi a sviluppare per i loro simili o al massimo per i loro capi. Ora che il frutto del lavoro
degli scienziati dell'informazione \`e destinato alle masse \`e arrivato il momento che gli informatici studino anche i principi fondamentali
dell'interazione uomo-macchina e uomo-computer.

Questo corso \`e una trattazione adattata per informatici delle teorie di human computer (HCI) e human machine interaction (HMI) [interazione
uomo-computer e interazione uomo-macchina]. 

Questo corso \`e ispirato alla teoria dell'interazione del Prof. Donald Norman ed in particolare, queste dispense sono in parte un adattamento didattico
del libro ``La caffettiera del masochista" di Donald Norman, pubblicato in Italia da Giunti e disponibile in lingua originale come ``The design of
everyday things, D. Norman". Consiglio di acquistare il libro per avere una trattazione con taglio più narrativo e sicuramente più esteso ed
approfondito di quanto qui riportato.

Nel corso verranno trattati anche aspetti relativi all'Internet delle cose e alle interazioni con robot e altri sistemi ``smart". Questi aspetti
legati all'interazione con oggetti smart sono anch'essi ispirati agli studi di Norman e sono ampiamente trattati nel libro
``Il Computer Invisibile, D. Norman", pubblicato in Italia da Apogeo.

L'obiettivo di questo corso \`e quindi quello di fornire agli studenti del corso di laurea in Informatica gli strumenti necessari a comprendere e
gestire il processo di sviluppo delle interfacce e dei prodotti interattivi. Questo corso ambisce quindi a spostare l'informatico dal suo tipico
ruolo di sviluppatore per farlo diventare un progettista non solo del ``codice" ma del prodotto nel suo insieme.

Nel corso parler\`o spesso di ``prodotto" (nei termini di questo corso, viene definito come qualsiasi cosa un utente usi), ``business",
``acquisto" e altri termini legati al mondo della vendita, dell'economia e del mercato. Questo perch\'e l'informatico deve a mio parere sviluppare
una consapevolezza fondamentale per il suo lavoro:

\vspace{\baselineskip}
\textbf{un prodotto che nessuno compra \`e un prodotto inutile}.
\vspace{\baselineskip}

Non importa quanto geniale sia il codice che avete scritto o rivoluzionario il sistema che avete implementato, se non vi curerete di far s\`i che questo
artefatto venga apprezzato (e quindi utilizzato dalle persone) la vostra creazione, geniale o stupida che sia, morir\`a dentro il vostro computer.

Per comprendere appieno la definizione di prodotto, si prende come esempio ``Google" (motore di ricerca): \`e gratuito, \`e un prodotto che viene
usato come servizio e ha una certa strategia di remunerazione.

\vfill
\textit{Queste dispense derivano dagli appunti di Simone Pepi e Francesco Iannelli pubblicati su \href{https://github.com/unipi-notes/HMI_Notes-2019-20}
{GitHub} e relativi all' A.A 19/20}.

\textbf{Queste dispense sono ad esclusivo uso degli studenti del corso di Programmazione di Interfacce dell'Universit\`a di Pisa. \`E
vietata la divulgazione, copia o riproduzione in qualsiasi forma}.
 %introduzione
\include{src/capitoli/capitolo_2} %Progettare l'Interazione fra Uomo e Macchina
\include{src/capitoli/capitolo_3} %Human Centered Design
\include{src/capitoli/capitolo_6} %Progettazione delle interfacce
\chapter{Principi Fondamentali dell'Interazione}
Un buon design produce un'esperienza piacevole! Ai tecnici non piace molto la parola \textit{esperienza} poiché la considerano troppo soggettiva. Ma se si interrogasse un ingegnere sulla sua automobile preferita descriverà modello e finiture, la sensazione di potenza nell'accelerazione, la maneggevolezza del cambio e dello sterzo.

L'esperienza è cruciale poiché determina \textbf{la tonalità del ricordo} che conserviamo delle interazioni con gli oggetti.

Quando la tecnologia si comporta in maniera inaspettata, gi utenti provano confusione, frustrazione e rabbia: tutte \textbf{emozioni negative}. Se invece comprendono il comportamento della tecnologia, hanno una sensazione di controllo, bravura e persino orgoglio: \textbf{emozioni positive}.

\textbf{Cognizione ed emozione sono profondamente legate}: se non si mette l'utente in un \textit{mood} positivo egli farà più fatica ad usare l'interfaccia. Più l'utente è arrabbiato e frustrato meno è predisposto riutilizzare il prodotto e ad impararne l'uso.

La \textbf{visibilità} o \textbf{discoverability} di un prodotto è il grado di facilità con cui un utente \textbf{scopre cosa fa, come funziona e che tipo di azioni sono possibili}. Tale visibilità è il risultato dell'applicazione di cinque concetti psicologici fondamentali: \textbf{affordance}, \textbf{significante}, \textbf{vincolo}, \textbf{mapping} e \textbf{feedback}. C'è anche un sesto principio, forse il più importante di tutti: \textbf{il modello concettuale del sistema}. Si analizzeranno di seguito:

\section{Affordance}
Il termine affordance, letteralmente \textit{invito}, indica la relazione fra un oggetto fisico e una persona, particolarmente la relazione fra le proprietà dell'oggetto e la capacità dell'utente di determinare in che modo tale oggetto può essere usato.

\textit{Una sedia appare fatta apposta per sostenere qualcosa quindi \textbf{invita} alla seduta. La maggior parte delle sedie è abbastanza leggera da poter essere sollevata e spostata da una singola persona possiamo dire che "
	invita o permette il trasporto, ma alcune sedie sono così pesanti da richiedere l'intervento di più persone. Se però un certo gruppo di individui non ha la forza di sollevare una sedia, per loro la sedia non presenta l'affordance di essere sollevata e trasportata}.

Un'affordance \textbf{non è una proprietà, è una relazione tra un oggetto e una persona}, dipende quindi dalle proprietà sia dell'oggetto che della persona stessa.

Si può anche parlare di \textbf{anti-affordance} nel concetto di \textbf{prevenzione dell'interazione}. Un ottimo esempio sono gli spunzoni per evitare che piccioni o altri tipi di volatili si posino in un cornicione: prevengono l'affordance di sedersi che il cornicione ha verso i piccioni.
Le affordance e le anti-affordance \textbf{devono essere discoverable e perceivable}.

\pagebreak

Questo fatto non è scontato: il vetro, famoso per la sua relativa invisibilità, occulta l'anti-affordance di precludere il passaggio.

Se uno di questi inviti o impedimenti all'uso non è percepibile c'è bisogno di qualche mezzo per segnalarne la presenza: il \textbf{significante}.

\textbf{\underline{N.B.}} È assolutamente sbagliato dire "\textit{metto un affordance}". È corretto dire "\textit{metto un significante}" ma solo se un'affordance è già presente.

\section{Significante}
I progettisti hanno dei problemi pratici: hanno bisogno di sapere come rendere comprensibili gli oggetti che creano. Lavorando sulla grafica degli schermi elettronici, dovevano trovare il modo di indicare quali parti potevano essere sfiorate, battute, fatte scivolare in sù, in giù o di lato, azioni che si potevano eseguire con le dita, con lo stilo o con il mouse.

Un significante è quindi \textbf{un modo per indicare dove effettuare un'azione}, data un'affordance che determina quali azioni sono possibili.

\begin{figure}[!h]
	\centering
	\includegraphics[scale = 0.7]{"immagini/Affordance vs Signifier"}
\end{figure}
\begin{itemize}
	\item \textbf{Affordance}: \textit{cosa si può fare?} \textit{quale azione è possibile compiere?}
	\item \textbf{Signifier}: \textit{dove è possibile fare l'azione?}
\end{itemize}
Molto spesso i significanti \textbf{sono indispensabili} poiché la maggior parte delle affordance sono invisibili. Per fare un esempio basti pensare alle porte scorrevoli: se i cardini non sono visibili, quando vede la maniglia la prima azione che una persona tenta di fare è quella di spingere o di tirare la porta, ma essa non si muoverà, è quindi necessario mettere un significante (e.g. un cartello o una scritta) che indica quale azione è necessaria per aprire la porta.

I significanti posso essere:
\begin{itemize}
	\item \textbf{Voluti o intenzionali}: come un'etichetta, una stringa o un'icona.
	\item \textbf{Accidentali o non intenzionali}: come ad esempio un sentiero tracciato da persone che camminano attraverso un campo o delle persone in fila alla stazione.
\end{itemize}
Nel design \textbf{i significanti sono molto più importanti delle affordances}, perchè comunicano come usare il prodotto o l'interfaccia. Ma come si può associare l'affordance e il significante ad azioni reali? Nella maggior parte dei casi tramite \textbf{convenzioni}. La comprensione di un'affordance percepita è dovuta anche alle convenzioni culturali.

\pagebreak

\begin{figure}[!h]
	\centering
	\includegraphics[scale = 0.7]{immagini/sign.png}
	\caption{La scritta pull è un significante, data l'affordance della porta di essere spinta o tirata.}
\end{figure}
\begin{figure}[!h]
	\centering
	\includegraphics[scale = 0.7]{immagini/sign1.png}
	\caption{Le persone che aspettano il treno sono un esempio di significante sociale.}
\end{figure}

\section{Mapping}
\textbf{Mapping} è un termine tecnico, ripreso dalla matematica, che indica la relazione fra gli elementi di due insiemi.

Il concetto di \textbf{mapping} è di grande importanza nella progettazione di interfacce, in particolare nel \textbf{posizionamento dei significanti}. La disposizione dei significanti può comunicare di più circa l'interfaccia e circa le sue funzionalità. Infatti se si sfrutta una corrispondenza spaziale fra la collocazione dei comandi e quella dei dispositivi comandati, risulta molto più facile capire come usarli.

Il modo migliore per \textit{fare} mapping è quello \textbf{naturale}, perché è un'attività in cui il cervello umano è molto bravo, i bambini imparano a fare mapping fin dai primi anni di vita. È da tenere presente che il concetto di \textbf{naturale} è ben diverso dal concetto di \textbf{universale}, poiché ci possono essere molti mappings che sembrano naturali ma che in realtà sono specifici a una cerchia di culture.

\begin{figure}[!h]
	\centering
	\includegraphics[scale = 0.7]{immagini/mapping.png}
	\caption{Mapping cattivo e mapping buono.}
\end{figure}

\pagebreak

\section{Feedback}
Il feedback è la comunicazione del risultato di un'azione, è una risposta che l'interfaccia dà all'utente.

Il feedback \textbf{deve essere immediato}, anche un ritardo di un decimo di secondo può essere troppo, se il ritardo è troppo lungo l'utente potrebbe rinunciare all'attività che stava compiendo e passare ad altro o addirittura non riuscire a comprendere l'origine del feedback stesso.

\textbf{Deve essere informativo}, non deve portare con sè troppa informazione, ma deve assolvere al proprio obiettivo, deve far capire che un'azione è in corso o che è stato prodotto il risultato che ci si aspetta. Uno scarso feedback può essere peggio di nessun feedback, perché distrae, crea confusione e di conseguenza frustrazione nell'utente.

Un altro caratteristica importante è \textbf{la semplicità}, un feedback non deve essere pedante: troppi annunci o segnali portano le persone ad ignorarli perdendo anche i feedback cruciali e veramente importanti. Il feedback deve essere \textbf{essenziale} e mantenere l'ambiente calmo e tranquillo.

\section{Modello concettuale}
Un modello concettuale è una descrizione altamente semplificata delle funzionalità di un sistema, non deve essere completa o accurata ma \textbf{utile}. I file, le cartelle e le icone che si vedono sullo schermo del computer aiutano le persone a crearsi un modello concettuale dei dati in memoria o delle applicazioni disponibili. In realtà il computer non contiene fascicoli o cartelle: esse sono solo concettualizzazioni ideate per facilitarne l'uso.

\textbf{I modelli semplificati sono preziosi e utili fintanto che le ipotesi che li supportano sono vere.} Nel cloud storage i file sembrano essere sul dispositivo, ma in molti casi il materiale è in realtà nel cloud. Il modello concettuale veicolato è quello di un archivio disponibile sui dispositivi degli utenti. Questo modello semplificato è utile e basta per il normale utilizzo, ma se il collegamento dei servizi si interrompe, può nascere confusione nell'utente: l'informazione è sempre presente sullo schermo, ma egli non può più salvarla o recuperare altri dati, cosa inspiegabile in relazione al modello concettuale precedentemente veicolato.

Il modello concettuale esprime \textbf{come il designer vuole che l'utente percepisca il prodotto}, sarebbe, in un certo senso, l'ambizione di progettare e comprendere la UX. Una volta che i progettisti hanno pensato e progettato il modello concettuale si implementa l'interfaccia, in modo che il modello concettuale venga veicolato all'utente tramite affordances, significanti e mapping presenti su di essa.

Quando una persona interagisce con il sistema o con il prodotto sviluppa un suo modello mentale. Un \textbf{modello mentale} è un modello concettuale nella mente dell'utente che rappresenta il modo in cui, secondo lui, funzionano le cose. Non solo persone diverse possono avere modelli mentali diversi dello stesso oggetto, ma la stessa persona può avere molteplici modelli, pertinenti ciascuno ad un aspetto diverso del suo funzionamento, e persino contraddittori gli uni con gli altri.

\textbf{Più è grande la differenza tra il modello mentale e quello concettuale, più l'utente farà fatica ad usare il sistema.}

L'ideale è che l'utente apprenda un modello concettuale giusto \textbf{direttamente dal device che utilizza}, senza andare a leggere manuali o istruzioni o, peggio ancora, che gli venga trasmesso da terzi. La comprensione di un dispositivo tramite passaparola porta all'\textbf{effetto del telefono senza fili}: l'interpretazione cambia da persona a persona. Per questo vi è necessità che il modello concettuale trasmesso dal prodotto sia pressoché unico in relazione a quello mentale che l'utente si costruisce. In questo contesto vale l'affermazione \textit{less is more} secondo cui se una feature è difficile da veicolare allora è meglio non implementarla.

\pagebreak

\section{Immagine di Sistema}
Le persone si creano di continuo, attraverso l'esperienza, l'addestramento e l'istruzione, modelli mentali di sé, degli altri, dell'ambiente e degli oggetti con cui esse interagiscono.

Questi modelli servono agli uomini da guida per realizzare i loro scopi e comprendere il mondo in cui vivono.

Come fanno gli uomini a formarsi un modello concettuale adeguato dei dispositivi che utilizzano? Non potendo parlare con il progettista, essi si basano su tutta l'informazione accessibile: l'aspetto dell'apparecchio, cosa hanno imparato dall'uso di oggetti simili in passato, cosa comunicano le pubblicità, i venditori, i pieghevoli illustrativi, il sito web e il libretto di istruzioni.

\textbf{L'insieme di tutta questa informazione è l'immagine di sistema}.

\begin{figure}[!h]
	\centering
	\includegraphics[scale = 0.75]{"immagini/Immagine di Sistema"}
\end{figure}

Come illustrato nella figura, il progettista e l'utilizzatore finale del prodotto costituiscono i vertici scollegati di un triangolo. Il vertice del triangolo più a destra è occupato dal modello concettuale del progettista, \textbf{cioè dalla sua concezione del prodotto in questione}.

Una volta commercializzato, il prodotto si stacca dal progettista: lo vediamo infatti isolato nel secondo vertice del triangolo.

\textbf{L'immagine di sistema è tutto ciò che si percepisce dalla struttura fisica prodotta, completa di documentazione, istruzioni,significanti e ogni informazione accessibile dal sito web o dal servizio di assistenza clienti}.

Il modello concettuale dell'utente deriva dall'immagine di sistema, mediante l'interazione con il prodotto, mediante letture, ricerche online e manuali. Il progettista si aspetta che il modello concettuale dell'utente coincida col suo, ma, non essendoci comunicazione diretta fra lui e l'utente, tutto il peso della comunicazione grava invece sull'immagine di sistema.

Questo spiega perché la comunicazione è un aspetto importante del buon design. \textbf{ Per quanto sia geniale il prodotto, se la gente non riesce ad usarlo l'accoglienza sarà cattiva}. Tocca al progettista fornire le informazioni adeguate per rendere il prodotto comprensibile ed usabile. Quel che più conta è presentare un modello concettuale capace di guidare l'utente quando le cose non vanno come dovrebbero.

Un buon modello concettuale è la chiave per avere prodotti comprensibili, di facile uso e gradevoli, e una buona comunicazione è importantissima per ottenere buoni modelli concettuali.

\pagebreak

 %Principi Fondamentali dell'Interazione
\include{src/capitoli/capitolo_8} %Constraints, Discoverability e Feedback
%\chapter{How do people do things}
\textit{È facile imparare alcune azioni elementari per far funzionare un dispositivo tecnico. Ma cosa succede se le cose non vanno come dovrebbero?
Come può l'utente accorgersene, e scoprire cosa fare? }
\vspace{\baselineskip}

Per chiarire meglio tutto questo è bene soffermarsi prima sulla psicologia umana e sui modi con cui gli uomini scelgono e valutano
le proprie azioni. Fatto ciò si passerà a esaminare il ruolo della cognizione e delle emozioni in tale processo: il piacere quando
le cose funzionano senza intoppi e la frustrazione quando le aspettative iniziali degli utenti non sono realizzate.

\section{I Golfi dell'Esecuzione e della Valutazione}
Quando usiamo un oggetto, ci si trova davanti due golfi: il \textbf{golfo dell'esecuzione}, nel quale si cerca di indovinare cosa fare,
e il \textbf{golfo della valutazione}, in cui ci si sforza di capire cosa è successo. Il compito del progettista è quello di aiutare
gli utenti a superare i due golfi e renderli il meno profondi possibili.

\begin{figure}[!h]
	\centering
	\includegraphics[scale=0.73]{immagini/Golfi}
	\caption{Golfo dell'esecuzione e golfo della valutazione.}
\end{figure}

Il \textbf{golfo della valutazione} corrisponde allo sforzo necessario per interpretare lo stato fisico del dispositivo e capire fino a
che punto sono state realizzate le aspettative e le intenzioni iniziali. Il golfo è stretto quando il dispositivo fornisce informazioni
sul proprio stato, in una forma facile da cogliere e interpretare.

\pagebreak

\textit{Quali sono gli elementi progettuali più importanti per superare il golfo della valutazione?}
\vspace{\baselineskip}

Il \textbf{feedback} e un \textbf{modello concettuale adeguato}.

\vspace{\baselineskip}
\textit{Quali sono gli elementi progettuali più importanti per superare il golfo dell'esecuzione?}
\vspace{\baselineskip}

\textbf{Significanti}, \textbf{constraints}, \textbf{buon mapping} e un modello concettuale adeguato.
\vspace{\baselineskip}

Entrambi i golfi sono presenti in molti apparati. Si incontrano spesso difficoltà, puntualmente liquidate accusando sè stessi. Di fronte
alle difficoltà nell'uso di congegni che ci si aspetta di saper usare si finisce inevitabilmente col pensare di essere stupidi. Oppure,
con dispositivi dall'aspetto più complicato, semplicemente ci si arrende, pensando di essere incapaci di utilizzarli. Queste spiegazioni
sono entrambe sbagliate. \textbf{Le difficoltà hanno origine nel design, non nell'utente}.

\section{I sette stadi dell'azione}
Compiere un'azione implica due fasi: \textbf{esecuzione e valutazione degli effetti}, \textbf{fare e interpretare}. Sia l'esecuzione
che la valutazione richiedono che si capisca come funziona l'oggetto su cui si applica l'azione e quali risultati essa produce. Entrambe
le fasi influiscono sullo stato emotivo dell'utente.

Un progettista deve conciliare il compito che l'utente vorrebbe eseguire e tutte le possibili azioni fisiche che può compiere per
eseguirlo. Una volta che l'utente specifica quali azioni compiere, bisogna far in modo le esegua concretamente: ciò costituisce lo
\textbf{stadio dell'esecuzione}. \textbf{Identificato il goal, o scopo, l'utente discende attraverso i tre stadi dell'esecuzione: pianificare,
specificare ed eseguire}. \textbf{La valutazione si articola anch'essa in tre stadi: percepire, interpretazione, confrontare}.
Ecco così che si hanno i \textbf{sette stadi dell'azione}: uno per lo scopo, tre per l'esecuzione e tre per la valutazione.
\begin{figure}[!hb]
	\centering
	\includegraphics[scale=0.35]{"immagini/Sette stadi"}
	\caption{Sette stadi dell'azione}
\end{figure}
\begin{itemize}
	\itemsep-0.3em
	\item \textbf{Scopo}: definire l'obiettivo.
	\item \textbf{Progettare}: definire l'azione da eseguire.
	\item \textbf{Specificare}: costruire una sequenza d'azione.
	\item \textbf{Eseguire}: eseguire la sequenza specificata.
	\item \textbf{Percepire}: osservare lo stato del mondo.
	\item \textbf{Interpretare}: elaborare la percezione.
	\item \textbf{Confrontare}: rapportare il risultato allo scopo.
\end{itemize}

La maggior parte delle azioni non richiede che si percorrano tutti i sette stadi in sequenza, ma quasi nessuna attività si risolve tramite
un'azione singola.

Di solito si applicano numerose sequenze e l'intera attività può durare ore o giorni. Ci sono molteplici circuiti di feedback, con cui i risultati
di un'attività vengono usati per indirizzare l'utente verso altre, in cui uno scopo genera scopi accessori e ogni un progetto sotto-progetti. Ci sono
attività in cui lo scopo originario è addirittura dimenticato, scartato o riformulato.

I sette stadi offrono uno schema per sviluppare nuovi prodotti o servizi. I golfi dell'esecuzione e della valutazione sono i punti più ovvi da cui
partire, offrendo entrambi spunti per migliorare il prodotto. I progettisti devono svilupparne capacità di osservazione.

\section{I sette Principi Fondamentali della Progettazione}
Il modello a sette stadi del ciclo d'azione è un prezioso sussidio per il design, in quanto introduce una lista di domande fondamentali. In generale,
ogni stadio dell'azione richiede specifiche strategie progettuali, e, viceversa, presenta occasioni proprie di disastro.

Ne derivano dunque sette domande, a cui dovrebbe poter rispondere chiunque stia usando un determinato prodotto.

\begin{itemize}
	\itemsep-0.3em
	\item \textbf{Cosa voglio ottenere?}
	\item \textbf{Quali sono le sequenze d'azione alternative?}
	\item \textbf{Quale azione posso fare ora?}
	\item \textbf{Come faccio questa azione?}
	\item \textbf{Cosa è successo?}
	\item \textbf{Cosa significa?}
	\item \textbf{Va bene? Ho realizzato il mio scopo?}
\end{itemize}

\begin{figure}[!h]
	\centering
	\includegraphics[scale=0.65]{"immagini/Sette Domande"}
	\caption{Sette domande.}
\end{figure}

Il progettista ha la responsabilità di garantire che a ogni stadio dell'azione il prodotto fornisca l'informazione necessaria per proseguire correttamente.

L'informazione che serve a rispondere alle domande nelle fasi attuative è definita come \textbf{feedforward}.

L'informazione che aiuta a capire quello che è successo nella fasi percettive è definita invece come \textbf{feedback}.

Il \textbf{feedforward} si realizza mediante l'uso opportuno di significanti, vincoli e mapping, anche il modello concettuale ha un ruolo importante.

Il \textbf{feedback} è dato dall'immediato cambiamento di stato che il prodotto deve mostrare all'utente e, anche qui, una parte importante è svolta
dal modello concettuale.

Sia il \textbf{feedback} sia il \textbf{feedforward} devono presentarsi in una forma facilmente interpretabile da chi utilizza il sistema. La
presentazione deve corrispondere alla visione che le persone hanno dello scopo che vogliono realizzare e alle loro aspettative. L'informazione
erogata deve essere immediatamente comprensibile.

Dalle risposte relative ai sette stadi dell'azione si ricavano sette principi fondamentali del design:

\begin{itemize}
	\itemsep-0.3em
	\item \textbf{Visibilità}: è bene che sia facile scoprire immediatamente quali azioni sono possibili e qual è lo stato attuale del dispositivo.
	\item \textbf{Feedback}: è opportuno che ci sia un'informazione completa e continua riguardo ai risultati delle azioni e allo stato attuale del
	prodotto o del servizio. Dopo aver eseguito un'azione, deve essere facile determinare il risultato.
	\item \textbf{Modello Concettuale}: il design dovrebbe fornire tutta l'informazione necessaria per creare un buon modello concettuale del sistema,
	che favorisca la comprensione e la sensazione di controllo da parte dell'utente. Il modello concettuale potenzia sia la visibilità, sia la valutazione
	dei risultati.
	\item \textbf{Affordance}: è bene che le affordance siano fatte apposta per rendere possibili le azioni desiderate e impossibili quelle indesiderate.
	\item \textbf{Significanti}: un uso efficace dei significanti assicura la visibilità e la comprensibilità dei comandi.
	\item \textbf{Mapping}: è necessario che la relazione fra i comandi e le rispettive azioni obbedisca ai principi del buon mapping, sostenuto,
	per quanto possibile, dalla disposizione spaziale e dalla contiguità temporale.
	\item \textbf{Vincoli}: bisogna fornire vincoli fisici, logici, semantici e culturali, in modo tale da guidare l'azione e facilitandone
	l'interpretazione.
\end{itemize}
Questi sette principi sono mappati uno ad uno sugli stadi d'azione dell'utente.

La maggior parte delle azioni svolte quotidianamente, goal e intenzioni non sono davvero specificati: sono opportunistici anziché pianificati.
Le \textbf{azioni opportunistiche} sono quelle in cui il comportamento scaturisce delle circostanze; completiamo le attività del giorno e facciamo
certe cose a mano a mano che se ne presenta l'occasione invece che come conseguenza di un'attenta analisi e pianificazione.

Gli utenti in questi casi agiscono in maniera non controllata e quindi non prevedibile. È difficile fare buon design per queste situazioni,
anche attenendosi a tutti i principi esposti fino ad ora: l'utente che agisce in maniera opportunistica romperà in ogni caso questi schemi.

\section{Pensiero umano}
La mente umana è infatti molto complessa, non esiste un indicatore che separi pensiero subconscio (non ne siamo consapevoli, è ``nascosto", veloce,
poco costoso in termini energetici, procede per pattern matching) e pensiero conscio (è ``manuale", lento, elaborato, dettagliato, orientato all'analisi
e costoso in termini di energie, interviene quando non si riesce ad avere una corrispondenza).

Il pensiero conscio interviene per l'apprendimento ma, al termine della fase iniziale, pratica e studio producono quello che gli psicologi chiamano
``overlearning": una volta che una abilità è stata ``overlearned", i gesti verranno svolti senza sforzi e automaticamente. Un esempio è quando, guidando
una macchina, si vede qualcosa attraversare la strada: a questo punto, frenare è un gesto automatico, avverrà ancora prima di aver capito cosa stesse
veramente attraversando.

Come ci ricordiamo le esperienze se la maggior parte delle azioni sono in modalità subconscia? Lo facciamo attraverso la memoria esperienziale, i
ricordi sono come un dipinto in cui sono evidenziati gli elementi che abbiamo ritenuto importanti; per recuperare quelli meno importanti, cerchiamo
di ricordare quella esperienza. Distinguiamo due tipi di memoria: \textbf{memoria dichiarativa}, utilizzata per recuperare informazioni fattuali;
\textbf{memoria procedurale}, (utilizzata) per recuperare informazioni procedurali.

\begin{figure}[!h]
	\centering
	\includegraphics[scale=0.27]{immagini/subconscio-conscio.png}
\end{figure}

Un altro elemento determinante per il passaggio da un pensiero subconscio a uno conscio è lo stato emozionale.

\begin{figure}[!h]
	\centering
	\includegraphics[scale=0.75]{"immagini/Livelli di Processing"}
	\caption{Livelli di processing.}
\end{figure}

La divisione in conscio e subconscio può essere ulteriormente affinata andando a dividere il modo in cui il cervello fa processing in tre livelli
procedurali: viscerale (o rettiliano), comportamentale e riflessivo.

\begin{itemize}
	\itemsep-0.3em
	\item \textbf{Livello viscerale}: è il livello più elementare, permette di rispondere prontamente in maniera subconscia, senza consapevolezza o
	controllo cosciente.
	\item \textbf{Livello comportamentale}: è la sede delle abilità apprese durante circostanze più o meno simili a quelle attuali. Durante l'esecuzione,
	il livello comportamentale è guidato dalle aspettative, e durante l'attesa di conferma di tali aspettative è invece guidato dalle emozioni. Il livello
	comportamentale stabilisce in che modo si compie una determinata azione e in che modo si interpreta un determinato feedback.
	\item \textbf{Livello riflessivo}: è il livello della cognizione conscia, è qui che si sviluppa la comprensione profonda e hanno luogo il ragionamento
	e i processi decisionali. Qui fanno capo i livelli più alti di emotività: soddisfazione e orgoglio, ma anche frustrazione e senso di colpa.
\end{itemize}

\textbf{Veicolare informazioni all'utente mentre egli si trova nel livello riflessivo è estremamente efficace}. Al livello riflessivo il suo pensiero
è conscio e le emozioni che egli produce sono le più durature.

Gli stimoli riflessivi sono parte integrante del ricordo degli eventi, è importante quindi creare nell'utente ricordi positivi mentre egli è in questo
stadio dell'azione, perché tali ricordi sono i più duraturi.

Inoltre è la riflessione, intesa come pensiero cosciente, che induce a consigliare un prodotto e raccomandarne l'uso o magari a sconsigliarlo.

I tre livelli di elaborazione contribuiscono tutti insieme a determinare lo stato emotivo e cognitivo dell'utente. Funzioni riflessive di alto livello
possono mettere in moto emozioni più elementari e  queste, a loro volta, possono stimolare attività cognitive di tipo riflessivo. %How do people do things
%\include{src/capitoli/capitolo_4} %Metodi e Strumenti per l'Innovazione
%\chapter{Le interfacce utente}

Un'interfaccia è qualcosa che sta fra due facce. E' il punto di contatto fra due sistemi che tentano di comunicare. L'interfaccia serve quindi per
comunicare. Un'interfaccia può essere fisica (pulsanti), grafica (immagini a monitor) o di altre forme come per esempio un interprete, un traduttore
simultaneo o un mediatore culturale. 

Le interfacce possono far comunicare due macchine fra loro come nel caso del processore che tramite l'interfaccia USB scambia dati con la stampante.
Oppure possono far comunicare l’uomo con la macchina, come il cruscotto di un’auto, i cursori di un amplificatore, il rubinetto del lavandino, il
manubrio e i pedali della bicicletta.

Un'interfaccia utente è quindi sempre composta da due parti. Una di queste parti appartiene ad una persona l'altra ad uno strumento. Lo strumento è
ciò che compie l'azione, l’interfaccia è ciò che serve per permettere all'utente di guidare lo strumento nell'esecuzione dell'azione. Per esempio,
in un coltello la lama è lo strumento (che compie l'azione del tagliare), il manico è l’interfaccia che consente all'uomo di usare la lama senza
tagliarsi e quindi di guidare lo strumento in maniera soddisfacente. 

\begin{figure}[!h]
	\centering
	\includegraphics[scale=0.8]{immagini/interfaccia.png}
	\caption{Il manico è un esempio di interfaccia.}
\end{figure}


Quando parliamo di \textbf{User Interface o UI}, in italiano Interfaccia Utente, parliamo quindi dello spazio (non necessariamente fisico, il dominio
di una interazione ha una dimensionalità pari a quella dei nostri sensi) di un sistema dove avviene l'interazione fra uomo-macchina. Tipicamente,
si parla di UI in ambito informatico e tecnologico e quindi le interfacce utente sono comunemente identificate come sistemi atti a mettere in
comunicazione l'uomo con computer, sistemi informatici e oggetti intelligenti.

\vspace{\baselineskip}
\textit{User interfaces are a mapping from the sensory, cognitive, and social human world to these collections of functions exposed by a computer
program.}

[Amy J. Ko]
\vspace{\baselineskip}

L'obiettivo primario dell'interazione fra uomo e macchina è quello di consentire all'utente di controllare e far funzionare la macchina in modo efficace.
L'interfaccia deve quindi essere progettata per semplificare l'interazione fra l'uomo e la macchina rendendo così l'esperienza d'uso piacevole e
prolifica. L'interazione fra uomo e macchina deve sempre essere facile, efficiente e divertente così da massimizzare la User Experience del prodotto.

E' importante ricordare che l'uomo si è evoluto grazie alla sua capacità di adattamento che ha la sua massima espressione nel libero arbitrio e nella
capacità di prendere decisioni non necessariamente basate sulla logica ma piuttosto sulle sensazioni e intuizioni. Viceversa le macchine hanno un
comportamento puramente deterministico e pertanto non hanno nessuna capacità di adattamento. L'interfacci uomo macchina va quindi ad avere un ruolo
fondamentale nell'interazione fra le parti dal momento che abilita la comunicazione fra due realtà aventi principi e modalità di "funzionamento"
diametralmente opposte.

Un'interfaccia ben progettata consente all'utente di controllare l'apparato richiedendo uno sforzo fisico e cognitivo minimo. La buona interfaccia
massimizza inoltre la quantità di informazioni utili trasferite all'utente durante l'interazione evitando un sovraccarico informativo che provocherebbe
nell'utente confusione e quindi frustrazione.

\begin{figure}[!h]
	\centering
	\includegraphics[width=0.9\textwidth]{immagini/tesla.jpg}
	\caption{Cruscotto della Tesla model S (\href{https://pinxcars.com/2013-tesla-model-s-cockpit/}{\underline{Fonte}}.)}
\end{figure}

Per questo motivo, la progettazione di un'interfaccia è per definizione un'attività interdisciplinare che va oltre la programmazione grafica e
abbraccia la psicologia, le neuroscienze, il design e la fisica.

Le interfacce sono organizzabili secondo livelli. L' \textbf{HID o Human Interface Device} è la periferica grazie al quale l'utente interagisce con
il sistema; come ad esempio mouse, monitor, gamepad, ecc.. Lo \textbf{HMI o Human Machine Interface} è invece un concetto che astrae dall' HID.
Con HMI, si intende infatti, tutto il sistema di interazione uomo macchina che usa l'HID come elemento di contatto fisico con l'utente. Nel computer
per esempio, la HMI è il sistema mouse+cursore+finestre. Il mouse e il monitor sono HID.

Quando la macchina in questione è un computer, HMI diviene \textbf{HCI o Human Computer Interface}.

\section{Classificazione delle interfacce}
Le interfacce utente sono tipicamente organizzate sulla base dei sensi che utilizzano per stabilire l'interazione fra umano e macchina. Gli umani
possiedono cinque sensi (tatto, vista, udito, olfatto e gusto). Questo porta ad identificare cinque categorie di interfacce possibili, più una sesta
che è legata al cosidetto senso dell'equilibrio (balance in inglese) che però non è considerato un senso vero e proprio nella fisiologia umana.

Possiamo quindi organizzare le interfacce in 6 categorie:
\begin{itemize}
	\itemsep-0.3em
	\item \textbf{Tactile UI} (touch, tatto);
	\item \textbf{Visual UI} (sight, vista);
	\item \textbf{Auditory UI} (sound, udito);
	\item \textbf{Olfactory UI} (smell, olfatto);
	\item \textbf{Gustatory UI} (taste, gusto);
	\item \textbf{equilibrial UI} (balance, equilibrio).
\end{itemize}

La maggior parte delle interfacce utente utilizza però più di un senso umano per stabilire il collegamento. Le interfacce che usano più di un senso
sono dette \textbf{CUI o Composite User Interface}. Le più comuni e note CUI sono chiaramente le famose \textbf{GUI o Graphical User Interface},
le quali sono composte da interfacce grafiche (visual) e tattili (tactile). 

Se ad una GUI andiamo ad aggiungere anche il suono otteniamo una \textbf{MUI o Multimedia User Interface}.

Quindi quando ci si riferisce all'interfaccia di una app con il termine GUI spesso compiamo un errore perchè ormai la maggior parte dei dispositivi
informatici ha anche una sorgente sonora che è utilizzata durante l'interazione (feedback audio del touch sullo schermo, per esempio) e quindi ci
troviamo di fronte ad una MUI e non ad una GUI.


È bene sottolineare che \textbf{estendere le interfacce con più canali (sensi) non è sempre una buona idea}.
\begin{figure}[!h]
	\centering
	\includegraphics[scale=0.85]{immagini/flora_video.png}
	\caption{Esempio: video di Facebook (\href{https://www.facebook.com/Lastknight/posts/10158944882367053}{\underline{Fonte}}).}
\end{figure}

Prendiamo ad esempio i video di
Facebook, i video vengono riprodotti di default con l'audio disattivato per aumentare l'usabilità del sistema.
Gli ingegneri di Facebook si sono accorti infatti che la maggioranza delle persone che visualizzando i video, mutavano immediatamente il suono per
varie ragioni (e.g. privacy o utilizzo di Facebook in momenti non opportuni), quindi hanno reso questa opzione di default. Ovviamente se ragionassimo
in termini di capacità e possibilità dell'interfaccia sembrerebbe assurdo bloccare di default l'utilizzo di un canale.
Questo processo di analisi ha portato poi a far evolvere il mondo dei video online inserendo di default i sottotitoli. Siamo quindi in una situazione in
cui per aumentare l'usabilità del sistema se ne riducono le funzionalità (di default).

%Questo, oltre ad essere un ottimo esempio di MUI riprogettata in GUI, è anche un esempio di tecnica ideata per gli utenti disabili e riusata per
%far fruire il prodotto a quelle personas che lo utilizzano in momenti in cui non possono usufruire dell'audio.


\section{Categorizzare le CUI}
Le CUI possono essere categorizzare in tre diverse macrocategorie:

\begin{itemize}
	\itemsep-0.3em
	\item \textbf{Standard}: usano dispositivi standard come tastiere, mouse e monitor
	\item \textbf{Virtual}: Bloccano all'utente l'interazione con il mondo reale e creano un mondo virtuale che funge da interfaccia fra l'utente e
	la macchina.
	\item \textbf{Augmented}: Non bloccano all'utente la percezione del mondo reale ma la vanno ad arricchire. L'interfaccia è quindi un mix di
	contenuti reali e virtuali che vanno ad arricchire la realtà \textbf{espandendola}.
\end{itemize}

\begin{figure}[!h]
	\centering
	\includegraphics[width=\textwidth]{immagini/standard-virtual-interfaces.png}
	\caption{Tipi di interfaccia.}
\end{figure}

Le CUI possono essere anche \textbf{classificate tramite il numero di sensi che utilizzano}. Ad esempio, lo \textit{Smell-O-Vision} è una CUI
standard 3S, cioè è una normale interfaccia di tipo standard che nell'utilizzo coinvolge 3 sensi dell'utente (Visione, Udito e Olfatto). Se si
aggiungesse un quarto senso (per esempio le poltrone mobili dei cinema 4D) diventerebbe 4S.

Quando un'interfaccia utente interagisce con tutti i sensi umani viene chiamata \textbf{Qualia Interface} (il termine ``qualia" deriva
dalla \href{https://it.wikipedia.org/wiki/Qualia}{\underline{teoria filosofica dei qualia}}).

\begin{figure}[!h]
	\centering
	\includegraphics[scale=0.2]{immagini/react.jpg}
	\caption{Esempio di interfaccia aumentata 3S: Microsoft Reactable.}
\end{figure}

\section{Human Interface Devices}

Un ``human interface device" è un dispositivo informatico usato da umani e, tipicamente, prende input da umani; col termine HID si intendono sia i
\textbf{dispositivi fisici} sia il \textbf{protocollo USB-HID}.

\subsection*{Protocollo HID}

Il termine HID è stato coniato da Microsoft per permettere l'innovazione nell'ambito dei dispositivi di input e per semplificare il processo di
installazione di questi dispositivi; prima dell'introduzione dello standard, questi seguivano protocolli che dipendevano dal loro tipo
(mouse, tastiere, joystick etc.): nel caso di nuovi dispositivi, le opzioni erano adeguarsi a un protocollo già esistente o sviluppare
nuovi driver specifici. Invece, i dispositivi HID spediscono pacchetti che ne descrivono il tipo e che contengono dati di svariati tipi e formati.

Un singolo driver HID procede col parsing del pacchetto di dati e permette l'associazione dinamica di dati I/O con le funzionalità del sistema: è
possibile mandare dati come dispositivo standard e, per esempio, lasciare al sistema operativo il compito di decidere come utilizzarli. Il protocollo
ha dei limiti, ma i sistemi operativi moderni riconoscono dispositivi USB HID standard (mouse, tastiera) senza bisogno di driver specifici.

Nel protocollo HID si distinguono due entità:
\begin{itemize}
	\itemsep-0.3em
	\item \textbf{device}: interagisce direttamente con l'umano.
	\item \textbf{host}: comunica col device e riceve dati sulla base delle azioni eseguite dall'umano; i dati in output vanno dall'host al
	device all'umano.
\end{itemize}

Inoltre, il protocollo ha reso molto semplice la costruzione di nuovi dispositivi introducendo il concetto di \textbf{HID descriptor}:
\textbf{è un pacchetto standard che definisce la categoria di appartenenza e la struttura di dato del device} (NON ha bisogno di essere generato
dal dispositivo, può essere hard coded) e che viene inviato non appena il device viene collegato all'host.
Tipicamente, il device salva sulla ROM l'HID descriptor e non ha bisogno di capirlo; infatti, alcuni mouse e tastiere in commercio sono implementate
usando una CPU a 8-bit.

Il ruolo dell'host è più complesso poich\'e ha bisogno di ricevere l'HID descriptor dal device e farne il parsing prima di poter comunicare con questo:
ha bisogno della potenza computazionale necessaria per l'interpretazione del pacchetto.
Essendo chiaro che non tutti gli host potrebbero essere capaci di interpretare l'HID descriptor, HID descrive anche il ``boot protocol": prevedendo
pacchetti di formato predefinito, sono supportati solo alcuni device specifici (solo tastiera e mouse) con alcune feature specifiche.

HID è stato esteso a una serie di protocolli, si riportano alcuni esempi:
\begin{itemize}
	\itemsep-0.3em 
	\item \textbf{Bluetooth HID}: usato per mouse e tastiere connesse via Bluetooth.
	\item \textbf{Serial HID}: usato per telecomandi su Windows Media Center.
	\item \textbf{ZigBee input device}: ZigBee (RF4CE) supporta dispositivi HID attraverso il profilo ZigBee input device.
	\item \textbf{HID over I²C}: usato per dispositivi embedded su Microsoft Windows 8.
	\item \textbf{HOGP} (HID over GATT): usato per dispositivi HID connessi attraverso BLE.
\end{itemize}

\pagebreak
\subsection*{Periferiche HID}

Le periferiche HID sono organizzate in due categorie:
\begin{itemize}
	\itemsep-0.3em
	\item \textbf{di input}: basati su sensori, converte la realtà fisica in segnale elettrici. Un esempio di sensore è un microfono.
	\item \textbf{di output}: basati su attuatori, converte segnali elettrici in perturbazioni nel mondo. Un esempio di attuatore è un altoparlante.
\end{itemize}

Dispositivi di input e di output venivano tradizionalmente divisi in classi sulla base del tipo di input/output usato dall'HID, al giorno d'oggi questa
classificazione tende a decadere poich\'e la maggior parte dei dispositivi moderni usano più tecnologie; di seguito, le classi:
\begin{itemize}
	\itemsep-0.3em
	\item \textbf{testi e caratteri};
	\item \textbf{posizioni};
	\item \textbf{suoni};
	\item \textbf{immagini};
	\item \textbf{parametri ambientali};
	\item \textbf{posizione};
	\item \textbf{parametri fisiologici e biologici}.
\end{itemize} %Le interfacce utente
%\include{src/capitoli/capitolo_10}
%\chapter{Errore Umano}

La maggior parte degli incidenti industriali è causata da errore umano:
le stime si aggirano tra il 75\% e il 95\% del totale.

\vspace{\baselineskip}
\textit{Come è possibile che tante persone siano così incompetenti?}
\vspace{\baselineskip}

La risposta è semplice! Non lo sono, il problema è nella mal progettazione e nel cattivo design.

Si continuano a produrre dispositivi e software che richiedono, a chi li usa, di mantenere per ore un'attenzione ed una vigilanza
complete, oppure di memorizzare procedure arcaiche, confuse e usate di rado, magari anche una sola volta nella vita. Si costringono
le persone a stare in un ambiente monotono senza nulla da fare per ore e ore, salvo dovere improvvisamente rispondere con rapidità e
precisione. Le si sottopone ad un ambiente di lavoro complesso e sovraccarico, dove sono continuamente interrotte durante l'esecuzione
di compiti simultanei. L'\textbf{interruzione è una delle cause che più frequentemente portano all'errore umano}.

Uno dei più grandi \textbf{problemi} è l'\textbf{atteggiamento delle persone verso gli errori} commessi. Quando un errore causa perdite economiche o,
peggio ancora, danni alle persone, si istituisce una speciale commissione d'inchiesta, che quasi immancabilmente trova i colpevoli.

Il passo successivo è punirli con multe, il licenziamento o addirittura il carcere. Essi vengono incolpati e puniti o, nella migliore delle ipotesi,
incolpati e riaddestrati. Tutto questo però non risolve il problema: lo \textbf{stesso errore continuerà a presentarsi}.  Per evitare di incorrere
nuovamente nell'errore, quando esso viene commesso, è bene  studiarne le cause, per poi ridisegnare il prodotto o le procedure in modo che esso non
si ripeta o, se dovesse ripetersi, che i danni siano ridotti al minimo.

\section{Root Cause Analysis}
La \textbf{Root Cause Analysis} consiste nell'indagare l'incidente finché non si trova la singola causa che ne è l'origine. Ovvero il momento nel tempo
quando effettivamente qualcuno ha preso decisioni o eseguito azioni sbagliate e,
una volta fatto ciò, accertare da cosa è derivato lo sbaglio. Purtroppo, troppo spesso, il processo si ferma non appena si scopre che una persona ha agito
in maniera impropria, additandola come \textit{colpevole}.

Cercare di trovare la causa di un incidente suona bene, ma ha due difetti.

\begin{itemize}
	\itemsep-0.3em
	\item La maggior parte degli incidenti \textbf{non ha una sola causa}. Da qui il \textbf{modello a groviera degli incidenti} di James Reason.
	\item Solitamente l'analisi delle cause profonde si ferma non appena trovato un errore umano.
\end{itemize}

\begin{figure}[!ht]
	\centering
	\includegraphics[scale=0.55]{immagini/Groviera.png}
	\caption{Modello a groviera.}
\end{figure}

Se una macchina smette di funzionare per un guasto o un malfunzionamento si cerca di capire come mai si è rotta o cosa l'ha portata a guastarsi. È
opportuno fare lo stesso quando si scopre un errore umano: \textbf{individuarne le cause}.

Quando durante l'analisi delle cause profonde si incontra, nella concatenazione di cause ed effetti, un errore umano, \textbf{il lavoro è appena
cominciato}: bisogna capire \textbf{perché} l'errore \textbf{è accaduto e cosa si può fare per prevenirlo}.

\section{I cinque perchè}

L'analisi delle cause profonde mira a determinare la causa \textbf{prima di un evento}, non la causa immediata.

In Giappone da tempo si usa a questo scopo una procedura detta \textit{"dei \textbf{cinque perché}"} ideata da Sakichi Toyota e impiegata dalla Toyota
nell'ambito del sistema di controllo qualità dei suoi prodotti.

Fondamentalmente quindi quando si cerca la ragione di un evento non ci si deve fermare dopo averne trovata solo una, ma bisogna continuare ad indagare
fino a che non si trovano le \textbf{vere cause di fondo}.

Va ripetuta davvero cinque volte?

No, ma chiamarla \textit{procedura dei cinque perché} sottolinea la necessità di proseguire anche dopo aver trovano una causa apparente.

Vediamo un esempio: \textbf{il veicolo non si accende}.

\begin{itemize}
	\itemsep-0.3em
	\item \textbf{Perché?} La batteria è morta.
	\item \textbf{Perché?} L'alimentatore non funziona.
	\item \textbf{Perché?} La cinghia dell'alternatore non funziona.
	\item \textbf{Perché?} La cinghia dell'alternatore era ben oltre il suo tempo di servizio e non è stata sostituita.
	\item \textbf{Perché?} Il veicolo non è stato mantenuto secondo le tempistiche raccomandate.
\end{itemize}

Quando le persone sbagliano, bisogna \textbf{cambiare il sistema} in modo da evitare l'errore
e, se non è possibile eliminarlo del tutto, almeno fare in modo di ridurne gli effetti.

Se il sistema lascia sbagliare gli utenti è \textbf{mal progettato}, se il sistema induce all'errore, allora è \textbf{progettato malissimo}.

\vspace{\baselineskip}
\textit{Perchè le persone sbagliano?}
\vspace{\baselineskip}

Perchè il design si concentra sulle esigenze del sistema e delle macchine, non su quelle degli utenti. Le macchine hanno bisogno in genere di
comandi precisi, obbligando l'operatore a introdurre esatte informazioni numeriche. Gli esseri umani non sono adatti ad esercitare grande precisione
e commettono spesso errori quando devono digitare lunghe sequenze di numeri o lettere.

Gli umani sono creativi, curiosi, costruttivi, particolarmente bravi nel creare modi nuovi di fare le cose e nel cogliere nuove opportunità. Compiti
monotoni, ripetitivi e precisi contraddicono tali qualità e vi entrano in conflitto.

\section{Definizione di errore}
Si definisce \textbf{errore umano} ogni deviazione dal comportamento \textit{appropriato}. Il termine appropriato è da prendere con le pinze, perché
in molte circostanze si scopre quale fosse il comportamento giusto solo successivamente.

Generalmente comunque si chiama \textit{errore} ogni comportamento che si discosta da quello generalmente accettato come giusto o adeguato.
\textbf{Errore} è il termine generale per tutte le situazioni sbagliate. È possibile dividere gli errori in \textbf{due} classi:

\begin{itemize}
	\item \textbf{Lapsus o Slips}: si ha un lapsus quando s'intende eseguire un'azione e si finisce per eseguirne un'altra. Nel caso del lapsus, l'azione
	eseguita non è quella voluta. Ci sono \textbf{due tipi principali di lapsus}:
	\vspace{-3.5mm}
	\begin{itemize}
		\itemsep-0.3em
		\item \textbf{di azione}: si esegue una azione sbagliata. Per esempio, si versa il latte nel caffé e si mette la tazza in frigo.
		\item \textbf{di memoria}: si dimentica di eseguire una azione o di valutarne i risultati. Per esempio, si dimentica il fornello acceso
		dopo aver terminato la cottura.
	\end{itemize}
	\textbf{I lapsus si hanno nelle fasi attuative e percettive dell'azione}.
	\item \textbf{Mistakes}: si ha un mistake quando è sbagliato il goal o lo scopo: da quel momento in poi le azioni, anche
	se eseguite a puntino, fanno parte dell'errore essendo di per sé inappropriate, in quanto parte di un progetto sbagliato. In questo tipo di errore
	l'azione è corretta ma l'intenzione no.
	I mistakes si suddividono in:
	\vspace{-3.5mm}
	\begin{itemize}
		\itemsep-0.3em
		\item \textbf{rule-based}: la diagnosi della situazione è giusta, ma poi viene scelto un corso d'azione sbagliato, seguendo una regola
		operativa errata. Per esempio, un meccanico diagnostica un difetto nella batteria di una macchina, decide di non sostituire la batteria perché
		funziona al 50\% delle prestazioni attese.
		\item \textbf{knowledge-based}: la diagnosi della situazione è sbagliata. Per esempio, il peso del carburante viene misurato in libbre anziché
		in chilogrammi.
		\item \textbf{memory-lapse}: un passaggio viene dimenticato nel momento in cui si fissano gli obiettivi o si esegue una procedura o se ne
		valutano i risultati. Per esempio, un meccanico fallisce nella gestione di un errore perché salta un passaggio.
	\end{itemize}
	\textbf{I mistakes si hanno nelle fasi di pianificazione e valutazione dell'azione}.
\end{itemize}

\begin{figure}[h!]
	\begin{subfigure}{0.49 \linewidth} \centering
			\includegraphics[scale=0.215]{immagini/Errors.png}
	\end{subfigure}
	\begin{subfigure}{0.49 \linewidth} \centering
			\includegraphics[scale=0.32]{immagini/Errors1.png}
	\end{subfigure}
\end{figure}


\section{Prevenzione dell'errore}

\begin{flushleft}
	\textit{Non dovrebbe essere possibile che un semplice errore provochi un danno diffuso.}
\end{flushleft}

Ecco che cosa dovrebbe essere fatto in fase di prevenzione:

\begin{itemize}
	\itemsep-0.3em
	\item \textbf{Comprendere le cause dell'errore} per minimizzarne il ripresentarsi.
	\item Effettuare \textbf{controlli di sensibilità}, ovvero, chiedersi se le azioni superano il \textit{test del buon senso}.
	\item Rendere possibile \textbf{annullare le azioni} (undo) o rendere più difficile fare ciò che non può essere annullato (per esempio con uso di locks).
	\item \textbf{Rendere} più \textbf{semplice la scoperta e la comprensione degli errori} e semplificarne la risoluzione.
	\item Non trattare l'azione come errore, piuttosto \textbf{aiutare l'utente a compiere correttamente l'azione}.
\end{itemize}

I \textbf{novizi}, gli utenti base, coloro meno esperti del sistema cadono in \textbf{mistakes} poiché non hanno una base di
conoscenza adeguata e sufficientemente strutturata, viceversa, gli \textbf{utenti esperti} che usano il software o il sistema tutti i giorni e
che lo conoscono bene commettono più errori di tipo \textbf{lapsus} poiché tendono ad eseguire i compiti in maniera automatica, quasi istintiva,
affidandosi al controllo subconscio, mentre un
principiante è costretto a fare molta attenzione, cosicché incorre meno nei lapsus.

I \textbf{mistakes} nascono dalla scelta di scopi e piani d'azione inadeguati, oppure, in sede di valutazione, dal confronto errato tra
risultati e scopi. In altre parole dipendono da \textbf{informazioni ambigue o poco chiare sullo stato attuale del sistema e dalla mancanza di un
buon modello concettuale}.

Si esamineranno adesso quali possono essere le cause di errore e come è possibile prevenirle.

Le \textbf{interruzioni} sono una delle più grandi cause di errore, \textbf{soprattutto i lapsus}. Quando un'attività viene interrotta da qualche
evento, il costo in attenzione è molto maggiore della perdita di tempo causata dell'interruzione. Per riprendere il lavoro è necessario ricordare
precisamente il precedente stato dell'attività: quale era l'obiettivo, a che punto del ciclo dell'azione si era rimasti e quale era lo stato del sistema.

La maggior parte dei sistemi rende difficile la ripresa di un azione a seguito di un'interruzione. Tuttavia riducendo i passaggi dell'azione è possibile
diminuire il costo d'attenzione necessario per riprendere la concentrazione dopo esser stati interrotti.

Un'altra causa di errore sono i \textbf{feedback errati}: avvisi fastidiosi o non necessari che si presentano spesso durante l'uso di un sistema. Spesso
vengono silenziati, disattivati o ignorati, \textbf{facendo perdere di significato anche quelli utili per il raggiungimento dello scopo}.

Se si usano i feedback per segnalare errori ed essi sono stati disattivati dall'utente, egli cadrà in errore non conoscendone nemmeno il motivo.
\textbf{Avvisi e metodi di sicurezza vanno usati con cura e intelligenza}.

Un numero sempre maggiore di macchine e sistemi offrono informazioni attraverso l'uso di interfacce vocali , ma come tutti gli approcci anche questo
ha dei pro e dei contro. Da una parte consente di fornire informazioni precise, specialmente quando
l'attenzione visiva è diretta da qualche altra parte, ma se l'ambiente è rumoroso o se ci sono diversi avvisi vocali contemporaneamente, tali avvisi
possono non essere compresi o risultare addirittura fastidiosi.

\pagebreak

\begin{figure}[!h]
	\centering
	\includegraphics[scale=0.6]{immagini/Damn.png}
\end{figure}

Per prevenire errori è possibile quindi utilizzare:
\begin{itemize}
	\itemsep-0.3em
	\item \textbf{Constraints}: aggiungendo vincoli alle azioni. I sistemi elettronici hanno un'ampia selezione di metodi che possono essere usati per
	ridurre l'errore. Uno di questi può essere \textbf{segregare i controlli}, cosicché controlli confondibili tra loro vengano piazzati lontani l'uno
	dall'altro. Un altro è di \textbf{separare i moduli}, cosicché qualsiasi controllo non direttamente rilevante all'operazione corrente non sia visibile
	a schermo ma richieda uno sforzo extra per essere raggiunto.
	\item \textbf{Undo}: comando che annulla le operazioni effettuate dal precedente. I sistemi migliori hanno \textbf{più livelli di undoing} in modo
	tale da annullare intere sequenze di azioni.
	\item \textbf{Messaggi d'errore e di conferma}: molti sistemi cercano di prevenire l'errore chiedendo conferma prima di eseguire un comando,
	specialmente quando l'azione distruggerà qualcosa di importante. Tuttavia queste richieste sono spesso mal temporeggiate, perché \textbf{dopo aver
	richiesto un'operazione le persone sono solitamente certe di volerla compiere}. Un controllo migliore sarebbe visualizzare sia l'azione da compiere
	che l'oggetto interessato, con l'opzione annulla o prosegui.\textbf{ I messaggi di avviso sono sorprendentemente inefficaci contro gli errori}.

	\item  \textbf{Controlli di Sensibilità}: i sistemi elettronici presentano il vantaggio di poter controllare che l'operazione richiesta sia
	\textbf{sensibile} o \textbf{ragionevole}. Ad esempio verificare che l'importo indicato sia giusto, magari esponendo un avviso in caso di numeri
	eccessivamente grandi.
\end{itemize}

In estrema sintesi e ricollegandosi all'esempio della groviera, per ridurre gli errori si hanno le seguenti possibilità:

\begin{figure}[!h]
	\centering
	\includegraphics[scale=0.5]{immagini/Groviera.png}
\end{figure}

\begin{itemize}
	\item \textbf{Aumentare il numero di controlli} (le fette).
	\item \textbf{Migliorare il modello concettuale dell'utente} (ridurre il numero di buchi, o rendere più piccoli i buchi esistenti, magari con un
	modello concettuale minimale e dei constraints).
	\item \textbf{Allertare l'operatore umano quando diversi buchi si allineano}.
\end{itemize}
%\include{src/capitoli/capitolo_12}
%\include{src/capitoli/capitolo_13}
%\include{src/capitoli/capitolo_14}




\end{document}