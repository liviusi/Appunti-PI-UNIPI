\chapter{Introduzione}

Negli ultimi anni il ruolo degli informatici \`e decisamente cambiato. Per anni l'informatico \`e stato chiuso in uno sgabuzzino ad interagire in
solitudine con una tastiera, bevendo bibite gassate mentre qualcuno gli diceva che cosa serviva (o almeno era convinto di sapere cosa servisse)
all'azienda per crescere.

Poi sono arrivati Internet, gli smartphone, la banda larga e l'Internet delle cose (IOT) e tutto \`e cambiato. 

Oggi gli informatici sono diventati le nuove star! 

Il mondo \`e ormai tecnocratico e le nuove soluzioni informatiche e tecnologiche hanno la capacit\`a di mutare la vita delle persone e gli andamenti
dell'economia in tempi così veloci da far rabbrividire.

Tuttavia, l'informatico \`e rimasto (spesso) in termini di attitudine e di bagaglio culturale lo stesso di prima. Li abbiamo fatti uscire dagli
sgabuzzini, abbiamo messo loro una giacca sopra la maglietta di Star Wars e li abbiamo spediti sui palchi dei tecnoeventi a fare pitch.

\`E chiaro che le competenze tecniche siano il bagaglio fondamentale per un informatico, ma in un epoca in cui le scelte degli informatici hanno
la potenzialit\`a di cambiare la vita delle persone non si può più prescindere da far capire agli informatici che cosa accade ad una persona ``normale"
(un non informatico, un babbano) quando interagisce con un software o con un sistema tecnologico in generale. 

Per troppi anni gli informatici hanno potuto limitarsi a sviluppare per i loro simili o al massimo per i loro capi. Ora che il frutto del lavoro
degli scienziati dell'informazione \`e destinato alle masse \`e arrivato il momento che gli informatici studino anche i principi fondamentali
dell'interazione uomo-macchina e uomo-computer.

Questo corso \`e una trattazione adattata per informatici delle teorie di human computer (HCI) e human machine interaction (HMI) [interazione
uomo-computer e interazione uomo-macchina]. 

Questo corso \`e ispirato alla teoria dell'interazione del Prof. Donald Norman ed in particolare, queste dispense sono in parte un adattamento didattico
del libro ``La caffettiera del masochista" di Donald Norman, pubblicato in Italia da Giunti e disponibile in lingua originale come ``The design of
everyday things, D. Norman". Consiglio di acquistare il libro per avere una trattazione con taglio più narrativo e sicuramente più esteso ed
approfondito di quanto qui riportato.

Nel corso verranno trattati anche aspetti relativi all'Internet delle cose e alle interazioni con robot e altri sistemi ``smart". Questi aspetti
legati all'interazione con oggetti smart sono anch'essi ispirati agli studi di Norman e sono ampiamente trattati nel libro
``Il Computer Invisibile, D. Norman", pubblicato in Italia da Apogeo.

L'obiettivo di questo corso \`e quindi quello di fornire agli studenti del corso di laurea in Informatica gli strumenti necessari a comprendere e
gestire il processo di sviluppo delle interfacce e dei prodotti interattivi. Questo corso ambisce quindi a spostare l'informatico dal suo tipico
ruolo di sviluppatore per farlo diventare un progettista non solo del ``codice" ma del prodotto nel suo insieme.

Nel corso parler\`o spesso di ``prodotto" (nei termini di questo corso, viene definito come qualsiasi cosa un utente usi), ``business",
``acquisto" e altri termini legati al mondo della vendita, dell'economia e del mercato. Questo perch\'e l'informatico deve a mio parere sviluppare
una consapevolezza fondamentale per il suo lavoro:

\vspace{\baselineskip}
\textbf{un prodotto che nessuno compra \`e un prodotto inutile}.
\vspace{\baselineskip}

Non importa quanto geniale sia il codice che avete scritto o rivoluzionario il sistema che avete implementato, se non vi curerete di far s\`i che questo
artefatto venga apprezzato (e quindi utilizzato dalle persone) la vostra creazione, geniale o stupida che sia, morir\`a dentro il vostro computer.

Per comprendere appieno la definizione di prodotto, si prende come esempio ``Google" (motore di ricerca): \`e gratuito, \`e un prodotto che viene
usato come servizio e ha una certa strategia di remunerazione.

\vfill
\textit{Queste dispense derivano dagli appunti di Simone Pepi e Francesco Iannelli pubblicati su \href{https://github.com/unipi-notes/HMI_Notes-2019-20}
{\underline{GitHub}} e relativi all' A.A 19/20}.

\textbf{Queste dispense sono ad esclusivo uso degli studenti del corso di Programmazione di Interfacce dell'Universit\`a di Pisa. \`E
vietata la divulgazione, copia o riproduzione in qualsiasi forma}.
