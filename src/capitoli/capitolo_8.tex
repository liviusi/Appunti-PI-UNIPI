\chapter{Vincoli}
%\chapter{Constraints, Discoverability e Feedback}
\begin{flushleft}
	\textit{In che modo si riesce a capire una cosa che non abbiamo mai visto prima?}
\end{flushleft}

Non c'è altro da fare che combinare l'informazione presente nel mondo esterno con quella che si ha in testa.

L'insieme di conoscenze che si trovano nel mondo comprende le affordances, i significanti visibili, le corrispondenze fra quelle parti degli oggetti
che sembrano comandi o punti da manipolare, le azioni risultanti e i vincoli fisici che limitano ciò che è possibile fare.

La conoscenza si ha in mente comprende i modelli concettuali, i vincoli culturali, semantici e logici del comportamento, le analogie fra la situazione
attuale ed esperienze precedenti.

\begin{figure}[!h]
	\centering
	\includegraphics[scale = 0.6]{"immagini/Modellino Lego"}
	\caption{Un modellino lego.}
\end{figure}

Si prenda come esempio il modellino Lego presente in figura: ha 15 pezzi, solo alcuni specializzati, molti altri sono di grandezza e forma uguale ma di
colori diversi. Combinando i vincoli fisici con quelli culturali, semantici e logici si riesce a costruire il modellino senza istruzioni, mettendo ogni
pezzo nella sua giusta posizione.

Vincoli fisici limitano le parti che possono andare insieme, i vincoli culturali e semantici impongono restrizioni precise a tutti i pezzi restanti e se
rimane fuori qualche pezzo l'incastro è dettato dalla logica.

\textbf{I vincoli sono indizi potenti}, che limitano l'insieme delle azioni possibili. L'uso intelligente dei vincoli in sede di design permette alle
persone di decidere prontamente il giusto corso d'azione, anche in una situazione del tutto nuova.

È possibile categorizzare i vincoli in \textbf{quattro} classi:
\begin{itemize}
	\itemsep-0.3em
	\item \textbf{Vincoli fisici}: si affidano a proprietà del mondo fisico, senza alcun bisogno di istruzioni o di addestramento. Nell'esempio della
	motocicletta Lego ritroviamo questo vincolo nei pezzi che si incastrano solo in un determinato verso.
	\item \textbf{Vincoli culturali}: si affidano alle abitudini culturali, sociali, comportamentali che possono cambiare nel tempo (si intendono
	anche le convenzioni). Nell'esempio della motocicletta Lego ritroviamo questo vincolo nel saper determinare la collocazione delle
	luci: bianco all'anteriore e rosso al posteriore.
	\item \textbf{Vincoli semantici}: si affidano al significato della situazione per circoscrivere l'insieme delle azioni possibili, si basano sulla
	conoscenza della situazione e del mondo. Nel caso della motocicletta, c'è un'unica collocazione sensata per il motociclista, deve stare seduto
	guardando in avanti.
	\item \textbf{Vincoli logici}: dettati dalla semplice e pura logica umana. Se avanzasse un un solo pezzo per assemblare la motocicletta, grazie alla
	logica sapremo dove collocarlo nella sua giusta posizione.
\end{itemize}

Un buon designer può sfruttare questi vincoli per veicolare l'utente verso un modello mentale del prodotto che si avvicini il più possibile al modello
concettuale desiderato e in tal modo garantirgli una UX gradevole.

%\section{Vincoli e mapping}
Vincoli e mapping a volte si confondono tra di loro. Una serie di interruttori mappati in maniera opportuna infondono un vincolo logico che permette
all'utente di non sbagliare, si intuisce perfettamente cosa verrà azionato da quell'interruttore posto in quel determinato punto. \textbf{Mapping forti
possono diventare dei vincoli logici}.
Come già ribadito più volte, l'assenza di vincoli e mapping genera frustrazione poiché crea una interfaccia poco chiara e difficile da comprendere.
\begin{figure}[!h]
	\centering
	\includegraphics[scale=0.67]{immagini/Interruttori}
	\caption{Un interruttore per le luci di una stanza che imita la piantina del locale.}
\end{figure}

\section{Funzioni Obbliganti}
\textbf{Le funzioni obbliganti sono una forma di vincolo fisico}: consistono di situazioni in cui le azioni sono vincolate in modo che un passaggio
mancato impedisca di procedere al successivo.

Sono il caso estremo di vincoli atti ad impedire un comportamento inappropriato.

Non tutte le situazioni permettono l'uso di vincoli così forti, ma il principio generale è applicabile negli ambiti più diversi.

Si esaminano ora tre di questi metodi:
\begin{itemize}
	\item \textbf{Interlock}: obbliga a eseguire una serie di operazioni (n maggiore di 1) nella sequenza dovuta prima di avviare l'azione richiesta.
	Gli interlock sono usati soprattutto come sistemi di sicurezza nei macchinari idnustriali ma anche nel mondo software come per esempio nel caso dei
	sistemi Captcha.

	Nelle figure che seguono si possono trovare due esempi di interlocks; un trapano a colonna professionale richiede che l'utente che lo usa esegua
	una serie di azioni nell'ordine corretto: chiudere lo sportello di sicurezza, armare il sistema premendo l'apposito bottone, attivare il motore
	ruotando l'interruttore rosso; se una volta attivato il sistema si apre lo sportello, il motore si spegne.
	
	\begin{figure}[h!]
		\begin{subfigure}{0.49 \linewidth} \centering
			\includegraphics[scale=0.35]{immagini/cap.png}
			\caption{Esempio di interlock: captcha.}
		\end{subfigure}
		\begin{subfigure}{0.49 \linewidth} \centering
			\includegraphics[scale=0.2]{immagini/trapano.png}
			\caption{Esempio di interlock: trapano a colonna.}
		\end{subfigure}
	\end{figure}

	\item \textbf{Lock-in}: mantiene attiva una funzione impedendo che qualcuno la interrompa prematuramente. È usato molto in ambito informatico (e.g.
	ogni tentativo di uscita da un'applicazione senza salvare è prevenuto da un messaggio di allerta che chiede la conferma dell'intenzione, ``blocca
	dentro" l'utente). \textbf{Per finire un task si deve compiere un'azione.}
	
	\begin{figure}[!h]
		\centering
		\includegraphics[width=0.4\textwidth]{immagini/savedialog.png}
		\caption{Esempio di lock-in: save dialog.}
	\end{figure}


	\item \textbf{Lockout}: impedisce l'ingresso in uno spazio pericoloso o impedisce che succeda qualcosa. Può essere considerato l'opposto del lock-in.
	Due esempi di stampo informatico: gli \textbf{alert VM 18} che si possono trovare su alcuni siti, bloccare un utente che ha immesso troppe volte
	credenziali sbagliate. \textbf{Per accedere ad un task si deve compiere un'azione.}

	\begin{figure}[!h]
		\centering
		\includegraphics[width=0.5\textwidth]{immagini/lockout.png}
		\caption{Esempio di lockout: blocco di un utente.}
	\end{figure}

\end{itemize}

\section{Activity-Centered Control}
Il mapping spaziale dei comandi non sempre è il più opportuno.

In molti casi è meglio avere interruttori diversi per attività diverse: \textbf{comandi centrati sulle attività}.

Azionando un semplice comando si impostano una serie di oggetti per svolgere una determinata attività, senza comandarli uno ad uno. In molti auditorium
ci sono interruttori con indicazioni \textit{video}, \textit{computer}, \textit{piena luce} e \textit{lezione} che impostano il microfono, le luci della
sala, il proiettore e quant'altro.

Questo schema è eccellente in teoria e funziona particolarmente bene per utenti esperti o per alcuni software specifici, ma nella pratica è difficile
da realizzare bene, soprattutto è necessario valutare gli \textbf{imprevisti} e le possibili risoluzioni.

Per carità, il metodo è giusto, purché la gamma di attività sia scelta in modo da corrispondere alle situazioni reali. Ma anche in quel caso saranno
pur necessari dei comandi manuali, perché si presenteranno sempre esigenze inattese, che richiederanno una regolazione particolare dei dispositivi.

La Logitech ha prodotto una serie di \href{https://www.logitech.com/it-it/harmony-universal-remotes}{\underline{telecomandi universali}} completamente
progettati attorno al concetto degli Activity Centered Control; un altro esempio sono i comandi disponibili su alcuni assistenti vocali (``Alexa,
imposta modalità cinema").

\begin{figure}[!h]
	\centering
	\includegraphics[width=0.5\textwidth]{immagini/harmony.png}
	\caption{Il telecomando universale Harmony di Logitech consente di controllare vari dispositivi attraverso un modello di interazione basato sugli
	activity centered controls. Non si seleziona più il dispositivo che si vuol controllare ma l'attività che si vuol fare: Play Game, Watch TV, etc..}
\end{figure}

Fondamentale ribadire che Activity-Centered Control e Forcing Functions non sono uno dei principi dell'interazione (vincoli, feedback, affordance,
significanti, mapping, immagine di sistema, discoverability) ma applicazioni: sono un'evidenza di come strutture particolari di controllo
fisico o presenti su una interfaccia derivino dal loro utilizzo.