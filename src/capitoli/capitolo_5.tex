\chapter{Le interfacce utente}

Un'interfaccia è qualcosa che sta fra due facce. E' il punto di contatto fra due sistemi che tentano di comunicare. L'interfaccia serve quindi per
comunicare. Un'interfaccia può essere fisica (pulsanti), grafica (immagini a monitor) o di altre forme come per esempio un interprete, un traduttore
simultaneo o un mediatore culturale. 

Le interfacce possono far comunicare due macchine fra loro come nel caso del processore che tramite l'interfaccia USB scambia dati con la stampante.
Oppure possono far comunicare l’uomo con la macchina, come il cruscotto di un’auto, i cursori di un amplificatore, il rubinetto del lavandino, il
manubrio e i pedali della bicicletta.

Un'interfaccia utente è quindi sempre composta da due parti. Una di queste parti appartiene ad una persona l'altra ad uno strumento. Lo strumento è
ciò che compie l'azione, l’interfaccia è ciò che serve per permettere all'utente di guidare lo strumento nell'esecuzione dell'azione. Per esempio,
in un coltello la lama è lo strumento (che compie l'azione del tagliare), il manico è l’interfaccia che consente all'uomo di usare la lama senza
tagliarsi e quindi di guidare lo strumento in maniera soddisfacente. 

\begin{figure}[!h]
	\centering
	\includegraphics[scale=0.8]{immagini/interfaccia.png}
	\caption{Il manico è un esempio di interfaccia.}
\end{figure}


Quando parliamo di \textbf{User Interface o UI}, in italiano Interfaccia Utente, parliamo quindi dello spazio (non necessariamente fisico, il dominio
di una interazione ha una dimensionalità pari a quella dei nostri sensi) di un sistema dove avviene l'interazione fra uomo-macchina. Tipicamente,
si parla di UI in ambito informatico e tecnologico e quindi le interfacce utente sono comunemente identificate come sistemi atti a mettere in
comunicazione l'uomo con computer, sistemi informatici e oggetti intelligenti.

\vspace{\baselineskip}
\textit{User interfaces are a mapping from the sensory, cognitive, and social human world to these collections of functions exposed by a computer
program.}

[Amy J. Ko]
\vspace{\baselineskip}

L'obiettivo primario dell'interazione fra uomo e macchina è quello di consentire all'utente di controllare e far funzionare la macchina in modo efficace.
L'interfaccia deve quindi essere progettata per semplificare l'interazione fra l'uomo e la macchina rendendo così l'esperienza d'uso piacevole e
prolifica. L'interazione fra uomo e macchina deve sempre essere facile, efficiente e divertente così da massimizzare la User Experience del prodotto.

E' importante ricordare che l'uomo si è evoluto grazie alla sua capacità di adattamento che ha la sua massima espressione nel libero arbitrio e nella
capacità di prendere decisioni non necessariamente basate sulla logica ma piuttosto sulle sensazioni e intuizioni. Viceversa le macchine hanno un
comportamento puramente deterministico e pertanto non hanno nessuna capacità di adattamento. L'interfacci uomo macchina va quindi ad avere un ruolo
fondamentale nell'interazione fra le parti dal momento che abilita la comunicazione fra due realtà aventi principi e modalità di "funzionamento"
diametralmente opposte.

Un'interfaccia ben progettata consente all'utente di controllare l'apparato richiedendo uno sforzo fisico e cognitivo minimo. La buona interfaccia
massimizza inoltre la quantità di informazioni utili trasferite all'utente durante l'interazione evitando un sovraccarico informativo che provocherebbe
nell'utente confusione e quindi frustrazione.

\begin{figure}[!h]
	\centering
	\includegraphics[width=0.9\textwidth]{immagini/tesla.jpg}
	\caption{Cruscotto della Tesla model S (\href{https://pinxcars.com/2013-tesla-model-s-cockpit/}{\underline{Fonte}}.)}
\end{figure}

Per questo motivo, la progettazione di un'interfaccia è per definizione un'attività interdisciplinare che va oltre la programmazione grafica e
abbraccia la psicologia, le neuroscienze, il design e la fisica.

Le interfacce sono organizzabili secondo livelli. L' \textbf{HID o Human Interface Device} è la periferica grazie al quale l'utente interagisce con
il sistema; come ad esempio mouse, monitor, gamepad, ecc.. Lo \textbf{HMI o Human Machine Interface} è invece un concetto che astrae dall' HID.
Con HMI, si intende infatti, tutto il sistema di interazione uomo macchina che usa l'HID come elemento di contatto fisico con l'utente. Nel computer
per esempio, la HMI è il sistema mouse+cursore+finestre. Il mouse e il monitor sono HID.

Quando la macchina in questione è un computer, HMI diviene \textbf{HCI o Human Computer Interface}.

\section{Classificazione delle interfacce}
Le interfacce utente sono tipicamente organizzate sulla base dei sensi che utilizzano per stabilire l'interazione fra umano e macchina. Gli umani
possiedono cinque sensi (tatto, vista, udito, olfatto e gusto). Questo porta ad identificare cinque categorie di interfacce possibili, più una sesta
che è legata al cosidetto senso dell'equilibrio (balance in inglese) che però non è considerato un senso vero e proprio nella fisiologia umana.

Possiamo quindi organizzare le interfacce in 6 categorie:
\begin{itemize}
	\itemsep-0.3em
	\item \textbf{Tactile UI} (touch, tatto);
	\item \textbf{Visual UI} (sight, vista);
	\item \textbf{Auditory UI} (sound, udito);
	\item \textbf{Olfactory UI} (smell, olfatto);
	\item \textbf{Gustatory UI} (taste, gusto);
	\item \textbf{equilibrial UI} (balance, equilibrio).
\end{itemize}

La maggior parte delle interfacce utente utilizza però più di un senso umano per stabilire il collegamento. Le interfacce che usano più di un senso
sono dette \textbf{CUI o Composite User Interface}. Le più comuni e note CUI sono chiaramente le famose \textbf{GUI o Graphical User Interface},
le quali sono composte da interfacce grafiche (visual) e tattili (tactile). 

Se ad una GUI andiamo ad aggiungere anche il suono otteniamo una \textbf{MUI o Multimedia User Interface}.

Quindi quando ci si riferisce all'interfaccia di una app con il termine GUI spesso compiamo un errore perchè ormai la maggior parte dei dispositivi
informatici ha anche una sorgente sonora che è utilizzata durante l'interazione (feedback audio del touch sullo schermo, per esempio) e quindi ci
troviamo di fronte ad una MUI e non ad una GUI.


È bene sottolineare che \textbf{estendere le interfacce con più canali (sensi) non è sempre una buona idea}.
\begin{figure}[!h]
	\centering
	\includegraphics[scale=0.85]{immagini/flora_video.png}
	\caption{Esempio: video di Facebook (\href{https://www.facebook.com/Lastknight/posts/10158944882367053}{\underline{Fonte}}).}
\end{figure}

Prendiamo ad esempio i video di
Facebook, i video vengono riprodotti di default con l'audio disattivato per aumentare l'usabilità del sistema.
Gli ingegneri di Facebook si sono accorti infatti che la maggioranza delle persone che visualizzando i video, mutavano immediatamente il suono per
varie ragioni (e.g. privacy o utilizzo di Facebook in momenti non opportuni), quindi hanno reso questa opzione di default. Ovviamente se ragionassimo
in termini di capacità e possibilità dell'interfaccia sembrerebbe assurdo bloccare di default l'utilizzo di un canale.
Questo processo di analisi ha portato poi a far evolvere il mondo dei video online inserendo di default i sottotitoli. Siamo quindi in una situazione in
cui per aumentare l'usabilità del sistema se ne riducono le funzionalità (di default).

%Questo, oltre ad essere un ottimo esempio di MUI riprogettata in GUI, è anche un esempio di tecnica ideata per gli utenti disabili e riusata per
%far fruire il prodotto a quelle personas che lo utilizzano in momenti in cui non possono usufruire dell'audio.


\section{Categorizzare le CUI}
Le CUI possono essere categorizzare in tre diverse macrocategorie:

\begin{itemize}
	\itemsep-0.3em
	\item \textbf{Standard}: usano dispositivi standard come tastiere, mouse e monitor
	\item \textbf{Virtual}: Bloccano all'utente l'interazione con il mondo reale e creano un mondo virtuale che funge da interfaccia fra l'utente e
	la macchina.
	\item \textbf{Augmented}: Non bloccano all'utente la percezione del mondo reale ma la vanno ad arricchire. L'interfaccia è quindi un mix di
	contenuti reali e virtuali che vanno ad arricchire la realtà \textbf{espandendola}.
\end{itemize}

\begin{figure}[!h]
	\centering
	\includegraphics[width=\textwidth]{immagini/standard-virtual-interfaces.png}
	\caption{Tipi di interfaccia.}
\end{figure}

Le CUI possono essere anche \textbf{classificate tramite il numero di sensi che utilizzano}. Ad esempio, lo \textit{Smell-O-Vision} è una CUI
standard 3S, cioè è una normale interfaccia di tipo standard che nell'utilizzo coinvolge 3 sensi dell'utente (Visione, Udito e Olfatto). Se si
aggiungesse un quarto senso (per esempio le poltrone mobili dei cinema 4D) diventerebbe 4S.

Quando un'interfaccia utente interagisce con tutti i sensi umani viene chiamata \textbf{Qualia Interface} (il termine ``qualia" deriva
dalla \href{https://it.wikipedia.org/wiki/Qualia}{\underline{teoria filosofica dei qualia}}).

\begin{figure}[!h]
	\centering
	\includegraphics[scale=0.2]{immagini/react.jpg}
	\caption{Esempio di interfaccia aumentata 3S: Microsoft Reactable.}
\end{figure}

\section{Human Interface Devices}

Un ``human interface device" è un dispositivo informatico usato da umani e, tipicamente, prende input da umani; col termine HID si intendono sia i
\textbf{dispositivi fisici} sia il \textbf{protocollo USB-HID}.

\subsection*{Protocollo HID}

Il termine HID è stato coniato da Microsoft per permettere l'innovazione nell'ambito dei dispositivi di input e per semplificare il processo di
installazione di questi dispositivi; prima dell'introduzione dello standard, questi seguivano protocolli che dipendevano dal loro tipo
(mouse, tastiere, joystick etc.): nel caso di nuovi dispositivi, le opzioni erano adeguarsi a un protocollo già esistente o sviluppare
nuovi driver specifici. Invece, i dispositivi HID spediscono pacchetti che ne descrivono il tipo e che contengono dati di svariati tipi e formati.

Un singolo driver HID procede col parsing del pacchetto di dati e permette l'associazione dinamica di dati I/O con le funzionalità del sistema: è
possibile mandare dati come dispositivo standard e, per esempio, lasciare al sistema operativo il compito di decidere come utilizzarli. Il protocollo
ha dei limiti, ma i sistemi operativi moderni riconoscono dispositivi USB HID standard (mouse, tastiera) senza bisogno di driver specifici.

Nel protocollo HID si distinguono due entità:
\begin{itemize}
	\itemsep-0.3em
	\item \textbf{device}: interagisce direttamente con l'umano.
	\item \textbf{host}: comunica col device e riceve dati sulla base delle azioni eseguite dall'umano; i dati in output vanno dall'host al
	device all'umano.
\end{itemize}

Inoltre, il protocollo ha reso molto semplice la costruzione di nuovi dispositivi introducendo il concetto di \textbf{HID descriptor}:
\textbf{è un pacchetto standard che definisce la categoria di appartenenza e la struttura di dato del device} (NON ha bisogno di essere generato
dal dispositivo, può essere hard coded) e che viene inviato non appena il device viene collegato all'host.
Tipicamente, il device salva sulla ROM l'HID descriptor e non ha bisogno di capirlo; infatti, alcuni mouse e tastiere in commercio sono implementate
usando una CPU a 8-bit.

Il ruolo dell'host è più complesso poich\'e ha bisogno di ricevere l'HID descriptor dal device e farne il parsing prima di poter comunicare con questo:
ha bisogno della potenza computazionale necessaria per l'interpretazione del pacchetto.
Essendo chiaro che non tutti gli host potrebbero essere capaci di interpretare l'HID descriptor, HID descrive anche il ``boot protocol": prevedendo
pacchetti di formato predefinito, sono supportati solo alcuni device specifici (solo tastiera e mouse) con alcune feature specifiche.

HID è stato esteso a una serie di protocolli, si riportano alcuni esempi:
\begin{itemize}
	\itemsep-0.3em 
	\item \textbf{Bluetooth HID}: usato per mouse e tastiere connesse via Bluetooth.
	\item \textbf{Serial HID}: usato per telecomandi su Windows Media Center.
	\item \textbf{ZigBee input device}: ZigBee (RF4CE) supporta dispositivi HID attraverso il profilo ZigBee input device.
	\item \textbf{HID over I²C}: usato per dispositivi embedded su Microsoft Windows 8.
	\item \textbf{HOGP} (HID over GATT): usato per dispositivi HID connessi attraverso BLE.
\end{itemize}

\pagebreak
\subsection*{Periferiche HID}

Le periferiche HID sono organizzate in due categorie:
\begin{itemize}
	\itemsep-0.3em
	\item \textbf{di input}: basati su sensori, converte la realtà fisica in segnale elettrici. Un esempio di sensore è un microfono.
	\item \textbf{di output}: basati su attuatori, converte segnali elettrici in perturbazioni nel mondo. Un esempio di attuatore è un altoparlante.
\end{itemize}

Dispositivi di input e di output venivano tradizionalmente divisi in classi sulla base del tipo di input/output usato dall'HID, al giorno d'oggi questa
classificazione tende a decadere poich\'e la maggior parte dei dispositivi moderni usano più tecnologie; di seguito, le classi:
\begin{itemize}
	\itemsep-0.3em
	\item \textbf{testi e caratteri};
	\item \textbf{posizioni};
	\item \textbf{suoni};
	\item \textbf{immagini};
	\item \textbf{parametri ambientali};
	\item \textbf{posizione};
	\item \textbf{parametri fisiologici e biologici}.
\end{itemize}